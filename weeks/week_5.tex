\section{Week 5}

Content in Week 5 does not weigh as heavy as other contents, so the summary will be short.

\vspace{0.5cm}

\colorbox{black}{\textbf{\color{white}Boundary Conditions (BCs)}}
\begin{itemize}
    \item \textbf{\color{red}Boundary conditions} (a.k.a. the \textbf{\color{red}initial conditions}) are the \textbf{\color{blue}real driver} for CFD solutions because they strongly dictate the particular solutions obtained from the governing equations.
\end{itemize}

\colorbox{black}{\textbf{\color{white}Momentum Boundary Conditions}}
\begin{itemize}
    \item No-slip wall
    \begin{itemize}
        \item A BC on a solid surface for viscous flows which assumes \textbf{\color{teal}zero relative velocity} between the surface and the fluid immediately at the surface
        \item If the surface is stationary,
        \begin{equation*}
            u=v=w=0
        \end{equation*}
    \end{itemize}
    \item Inflow condition
    \begin{itemize}
        \item This BC is needed because solutions of the governing equations require at least one velocity component
        \item E.g. $u=f$ and $v=w=0$ at the inflow boundary ($f$ can be constant or velocity profile)
    \end{itemize}
    \item Outflow condition
    \begin{itemize}
        \item This BC needs to positioned at locations where the flow is \textbf{\color{red}unidirectional} and where \textbf{\color{blue}surface stresses taken known values}.
        \item In a fully-developed flow exiting the channel, there is no change in velocity component in the direction across the boundary
        \item To satisfy stress continuity, the shear forces along the surface are taken to be zero:
        \begin{equation*}
            \frac{\partial u}{\partial n} = \frac{\partial v}{\partial n} = \frac{\partial w}{\partial n} = 0
        \end{equation*}
    \end{itemize}
\end{itemize}


\colorbox{black}{\textbf{\color{white}Validation and Verification}}
\begin{itemize}
    \item  Verification
    \begin{itemize}
        \item Check whether the conceptual model is programmed correctly into a computer model
        \item Are we solving it numerically accurately?
        \item Provide evidence that the equations are solved right:
        \begin{itemize}
            \item Do we have Mesh Sensitivity / Grid Independence/Convergence? Can be examined via key functionals plots (functionals against number of elements) or by grid convergence index (GCI) which quantifies for the grid error
            \item How about grid quality? 
            \item Is the residue target (convergence criterion) sufficient?
            \item Is the spatial and temporal discretization sufficient? Are there enough grid points / enough timesteps?
            \item Is the chosen discretization scheme accurate?
        \end{itemize}
    \end{itemize}
    \item Validation
    \begin{itemize}
        \item Check whether the computer model reflects the actual physical phenomenon.
        \item Provide evidence the right model is solved:
        \begin{itemize}
            \item Are the BCs similar to the actual experimental conditions? Are the boundaries positioned in the right location (e.g. far field boundaries)?
            \item Is the geometry of bluff body correct?
            \item Is the right turbulence model being used?
            \item What is the error between the simulated solutions and the experimental data?
        \end{itemize}
    \end{itemize}
\end{itemize}

\colorbox{black}{\textbf{\color{white}Types of Errors}}
\begin{itemize}
    \item Acknowledged error
    \begin{itemize}
        \item Can be identified with strategic approaches (e.g. approximation, discretization)
    \end{itemize}
    \item Unacknowledged error
    \begin{itemize}
        \item Cannot be identified with strategic approaches (i.e. programming error) 
    \end{itemize}
    \item Truncation error
    \begin{itemize}
        \item When we approximate a PDE by a difference formula, we neglect the Higher Order Term (H.O.T.)
    \end{itemize}
    \item Discretization error
    \begin{itemize}
        \item Difference between exact solution values and those obtained with discretized equations
    \end{itemize}
    \item Round-off error
    \begin{itemize}
        \item Computers use a finite word length of the real numbers, which leads to rounding errors.
    \end{itemize}
\end{itemize}

\colorbox{black}{\textbf{\color{white}Numerical Concepts}}
\begin{itemize}
    \item Consistency
    \begin{itemize}
        \item If the algebraic equations can be shown to be equivalent to the original governing equations as the grid spacing and time steps tends to zero (i.e. $\Delta x \to 0$ and $\Delta t \to 0$), then the numerical model is consistent.
    \end{itemize}
    \item Convergence 
    \begin{itemize}
        \item The property of a numerical method to produce a solution which approaches the exact solution as the grid spacing tends to zero
    \end{itemize}
    \item Stability
    \begin{itemize}
        \item Associated with the damping of errors in the numerical scheme
    \end{itemize}
    \item Conservativeness
    \begin{itemize}
        \item Ensures global conservation of fluid properties through the domain
    \end{itemize}
    \item Boundedness
    \begin{itemize}
        \item Related to stability - ensures the solution is bounded by maximum and minimum values of the flow variables
    \end{itemize}
    \item Transportiveness
    \begin{itemize}
        \item Accounts for directionality of diffusion and convection flows
    \end{itemize}
\end{itemize}

