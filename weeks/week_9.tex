\section{Week 9}
\colorbox{black}{\textbf{\color{white}Finite Difference Approximation (FDA)}}

\begin{itemize}
    \item \textbf{\color{red}Taylor series expansion} are used to approximate the partial derivatives. 
    \item Truncation errors exist due to higher order terms are ignored. \textbf{\color{orange}Truncation errors are negligible if the system is stable}.
\end{itemize}

\colorbox{black}{\textbf{\color{white}Finite Difference Methods (FDM)}}
\begin{itemize}
    \item FDMs discretize the domain into a grid. 
    \item Variables are evaluated at the grid points.
    \item FDM is commonly applied to \textbf{\color{red}structured mesh} as it requires a mesh having a high degree of regularity
    \item Pros: FDM allows the use of higher-order differencing approximations which provide a higher degree of accuracy in the solution
    \item \colorbox{teal}{\textbf{\color{white}Approximations of 1st order derivative}}
    \begin{itemize}
        \item \textbf{\color{red}Central Difference}
        \begin{alignat*}{2}
            &\text{Space:} \qquad && \frac{\partial \phi}{\partial x} = \frac{\phi_{i+1,j}-\phi_{i-1,j}}{2\Delta x} + O(\Delta x^2) \\
            &\text{Time:}\qquad && \text{N/A}
        \end{alignat*}
        \item \textbf{\color{red}Forward Difference}
        \begin{alignat*}{2}
            &\text{Space:}\qquad &&\frac{\partial \phi}{\partial x} = \frac{\phi_{i+1,j}-\phi_{i,j}}{\Delta x} + O(\Delta x)\\
            &\text{Time:}\qquad && \frac{\partial \phi }{\partial t} = \frac{\phi_{i,j}^{n+1}-\phi_{i,j}^{n}}{\Delta t} + O(\Delta t)
        \end{alignat*}
        \item \textbf{\color{red}Backward Difference}
        \begin{alignat*}{2}
            &\text{Space:} \qquad && \frac{\partial \phi}{\partial x} = \frac{\phi_{i,j}-\phi_{i-1,j}}{\Delta x} + O(\Delta x) \\
            &\text{Time:} \qquad && \frac{\partial \phi}{\partial t} = \frac{\phi_{i,j}^{n}-\phi_{i,j}^{n-1}}{\Delta t} + O(\Delta t)
        \end{alignat*}
    \end{itemize}
    \item \colorbox{teal}{\textbf{\color{white}Approximation of 2nd order derivative}}
    \begin{itemize}
        \item \textbf{\color{red}Central Difference}
        \begin{alignat*}{2}
            &\text{Space:} \qquad && \frac{\partial^2 \phi}{\partial x^2} = \frac{\phi_{i+1,j}-2\phi_{i,j}+\phi_{i-1,j}}{\Delta x^2} + O(\Delta x^2)
        \end{alignat*}
    \end{itemize}
\end{itemize}

\colorbox{black}{\textbf{\color{white}Finite Volume Methods (FVM)}}

\begin{itemize}
    \item FVMs discretize the domain into a number of contiguous control volumes.
    \item Variables are evaluated at the centroids of CV, and interpolation (a.k.a discretization schemes) is used to obtain variables values at the control volume face. 
    \item FVM can be applied to both \textbf{\color{red}structured} and \textbf{\color{red}unstructured} mesh.
    \item Cons: higher-order ($>$2nd order) differencing approximations are more difficult to obtain.
    \item \colorbox{orange}{\textbf{\color{white}Approximation Schemes of Surface Variables}}
    \begin{figure}[H]
        \centering
        \includegraphics[width=1.0\linewidth]{images/summary_discretization_scheme.png}
    \end{figure}
    \begin{itemize}
        \item \textbf{\color{red}Central Difference}
        \begin{align*}
            \phi_e &= \frac{1}{2} (\phi_P + \phi_E) \\
            \text{where }\; \phi_e &= \text{value at face}\\
            \phi_P &= \text{value at owner cell} \\
            \phi_E &= \text{value at neighbor cell}
        \end{align*} 
        Alternatively, using subscript notations,
        \begin{align*}
            \phi_f &= \frac{1}{2} (\phi_i + \phi_{i+1}) \\
            \text{where }\; \phi_f &= \text{value at face}\\
            \phi_i &= \text{value at owner cell} \\
            \phi_{i+1} &= \text{value at neighbor cell}
        \end{align*} 
        \begin{itemize}
            \item Pros: 2nd order accuracy
            \item Cons: can become unstable; may give large undershoots or overshoots
        \end{itemize}
        \item \textbf{\color{red}Upwind Difference (1st order)}
        \begin{alignat*}{2}
            & \phi_f = \phi_{i} \qquad && F_e > 0 \; \text{(backward)} \\
            & \phi_f = \phi_{i+1} \qquad && F_e < 0 \; \text{(forward)}
        \end{alignat*}
        An example approximating the term $u \frac{\partial T}{\partial x}$:
        \begin{alignat*}{2}
            & \frac{\partial T}{\partial x} = \frac{T_{i}^{n}-T_{i-1}^{n}}{\Delta x} \qquad && \text{if }\; u > 0 \\
            & \frac{\partial T}{\partial x} = \frac{T_{i+1}^{n}-T_{i}^{n}}{\Delta x} \qquad && \text{if }\; u < 0 
        \end{alignat*}
    \end{itemize}
\end{itemize}