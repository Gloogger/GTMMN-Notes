\section{Week 3}
\colorbox{black}{\textbf{\color{white}Governing equations}}

\begin{itemize}
    \item Governing equations are mathematical statements of the \textbf{{\color{red}conservation laws of physics}}:
    \begin{itemize}
        \item \textbf{\color{orange}Conservation of Mass} (a.k.a. Continuity Equation)
        \begin{itemize}
            \item Mass is conserved for the fluid
        \end{itemize}
        \item \textbf{\color{orange}Conservation of Momentum} (a.k.a. Newton's 2nd Law / the Navier-Stokes Equation)
        \begin{itemize}
            \item The rate of change of momentum equals the sum of forces acting on the fluid
        \end{itemize}
        \item \textbf{\color{orange}Conservation of Energy} (a.k.a. 1st Law of Thermodynamics)
        \begin{itemize}
            \item The rate of change of energy equals the sum of rate of heat addition to and the rate of work done on the fluid
        \end{itemize}
        \item \textbf{\color{orange}Species Transport Equation} (a.k.a. Principal of Species Conservation)
    \end{itemize}
\end{itemize}

\colorbox{black}{\textbf{\color{white}Auxiliary / Completion Equations}}
\begin{itemize}
    \item Completion equations are relations needed to complete data to solve the \textbf{governing equations}
    \item Examples:
    \begin{itemize}
        \item Equations of State: Gas Law, Density Relations
        \item Constitutive Relations: Viscosity, Turbulence Models
        \item Fluid Property Relations: Conductivity, Specific Heat
        \item Initial / Boundary Conditions
        \item Equations specific to a given problem
    \end{itemize}
\end{itemize}

\colorbox{black}{\textbf{\color{white}Substantial Derivative}}
\begin{itemize}
    \item Substantial Derivative is also called \textbf{\color{orange}total derivative}, \textbf{\color{red}convective derivative}, \textbf{\color{blue}material derivative}, or \textbf{\color{teal}substantive derivative.}
    \begin{itemize}
        \item Substantial derivative $\frac{D\phi}{Dt}$ is the time rate of change following a moving fluid element
        \item Partial derivative $\frac{\partial \phi}{\partial t}$ is the time rate of change  at a fixed location
    \end{itemize}
    \item As a math concept, the substantial derivative is defined as
    \begin{equation*}
        \frac{D(\cdot)}{Dt} = \frac{\partial (\cdot)}{\partial t} + \bm{V}\cdot \frac{\partial (\cdot )}{\partial \bm{x}}
    \end{equation*}
    For example, if temperature $T$ is a function of time and spatial coordinates, then the material derivative of $T(t,x,y,z)$ is,
    \begin{align*}
        \frac{DT}{Dt} &= \frac{\partial T}{\partial t} + \bm{V} \cdot \nabla T \\
        &= \frac{\partial T}{\partial t} + u \frac{\partial T}{\partial x} + v \frac{\partial T}{\partial y} + w \frac{\partial T}{\partial z} \\
        &= \frac{\partial T}{\partial t} + \left(\frac{\partial T}{\partial x}\right)\left(\frac{\partial x}{\partial t}\right) + \left(\frac{\partial T}{\partial y}\right)\left(\frac{\partial y}{\partial t}\right)\\
        &+\left(\frac{\partial T}{\partial z}\right)\left(\frac{\partial z}{\partial t}\right) \\
        &\equiv \left.\left.\frac{d}{dt}\right[T(t,x,y,z)\right]
    \end{align*}
    \item For a general variable property \textbf{per unit mass} denoted as $\phi$, we have the \textbf{\color{orange}conservative form} of general variable property equation:
    \begin{equation*}
        \frac{D\phi }{D t} = \frac{\partial \phi}{\partial t} + u \frac{\partial \phi}{\partial x} + v \frac{\partial \phi}{\partial y} + w \frac{\partial \phi}{\partial z}
    \end{equation*}
    if $\phi$ is in \textbf{per unit volume}, we have the \textbf{\color{orange}non-conservative form}:
    \begin{equation*}
        \rho\frac{D\phi }{D t} = \rho \frac{\partial \phi}{\partial t} + \rho u \frac{\partial \phi}{\partial x} + \rho v \frac{\partial \phi}{\partial y} + \rho w \frac{\partial \phi}{\partial z}
    \end{equation*}
    \item Only the \textbf{\color{orange}non-conservative form} of the general variable property equation is used to derive the momentum theorem.
\end{itemize}

\colorbox{black}{\textbf{\color{white}Continuity Equation / Mass Conservation}}
\begin{itemize}
    \item Developed from the concept: 
    \begin{equation*}
        \frac{\partial m}{\partial t}= \sum_{\text{in}}\dot{m} - \sum_{\text{out}}\dot{m}
    \end{equation*}
    \item \colorbox{orange}{\textbf{\color{white}General Form:}}
    \begin{align*}
        &\frac{\partial \rho}{\partial t} = - \frac{\partial (\rho u)}{\partial x} - \frac{\partial (\rho v)}{\partial v} - \frac{\partial (\rho w)}{\partial z} \\
        \text{or }\; &\frac{\partial \rho}{\partial t} + \frac{\partial (\rho u)}{\partial x} + \frac{\partial (\rho v)}{\partial v} + \frac{\partial (\rho w)}{\partial z} = 0
    \end{align*}
    \item \colorbox{orange}{\textbf{\color{white}Vector Form:}}
    \begin{equation*}
        \frac{\partial \rho}{\partial t} + \nabla \cdot (\rho \bm{V}) = 0
    \end{equation*}
    \item \colorbox{teal}{\textbf{\color{white}Simplifications}}
    \begin{itemize}
        \item When the flow is \textbf{\color{red}steady state}, the partial derivative w.r.t. time is zero, leaving
        \begin{align*}
            {\color{red}\cancelto{0}{\color{black}\frac{\partial \rho}{\partial t}}} + \nabla \cdot (\rho \bm{V}) &= 0
        \end{align*}
        \item For an \textbf{\color{purple}incompressible fluid}, we have $\rho =$ constant. Hence, the derivative w.r.t. time is zero and $\rho$ can be factored out the partial derivatives w.r.t. space, leaving
        \begin{align*}
            \frac{\partial u}{\partial x} + \frac{\partial  v}{\partial y} + \frac{\partial w}{\partial z} &= 0 \\
            \nabla \cdot \bm{V} &= 0
        \end{align*}
    \end{itemize}
    \item \colorbox{teal}{\textbf{\color{white}Separation of Density \& Velocity Vectors}}
    \begin{itemize}
        \item Using the product rule, the general form of continuity equation becomes,
        \begin{align*}
            {\color{blue}\underbrace{\frac{\partial \rho}{\partial t} + \bm{V}\cdot \nabla \rho}_{:=\frac{D\rho}{Dt}}} + \rho (\nabla \cdot \bm{V}) &= 0
        \end{align*}
        \item If the material is incompressible, then $\rho$ cannot change, so $\frac{D\rho}{Dt}$ must be zero. Thus,
        \begin{equation*}
            \rho (\nabla \cdot \bm{V}) = 0
        \end{equation*}
    \end{itemize}
\end{itemize}



\colorbox{black}{\textbf{\color{white}Key Properties of Momentum Equation}}
\begin{itemize}
    \item A moving fluid element experiences three types of force: \textbf{\color{blue}body forces}, \textbf{\color{red}surface forces}, \textbf{external forces}.
    \item \textbf{\color{blue}Body forces} are due to \textbf{\color{teal}external source}, e.g. gravitational forces, magnetic or electrical forces.
    \item \textbf{\color{red}Surface forces} are due to surface stresses acting on the control volume, which include normal stress $\sigma_{xx}$ and tangential stresses $\tau_{yx}$ and $\tau_{zx}$. 
    \begin{itemize}
        \item Normal stresses are caused by pressure p
        \item Tangential stresses are caused by viscosity. For Newtonian fluids, by Newton's Law of Viscosity, the tangential stresses can be related to linear deformation by the \textbf{\color{orange}first dynamic viscosity constant $\mu$}, and to the volumetric deformation using the \textbf{\color{orange}second dynamic viscosity constant $\lambda$}:
        \begin{align*}
            \tau_{xx} &= {\color{red}2 \mu \frac{\partial u}{\partial x} }+ \lambda \left[ \frac{\partial u}{\partial x} + \frac{\partial v}{\partial y} + \frac{\partial w}{\partial z} \right] \\
            \tau_{yy} &= {\color{red}2 \mu \frac{\partial v}{\partial y}} + \lambda \left[ \frac{\partial u}{\partial x} + \frac{\partial v}{\partial y} + \frac{\partial w}{\partial z} \right] \\
            \tau_{zz} &= {\color{red}2 \mu \frac{\partial w}{\partial z} } + \lambda \left[ \frac{\partial u}{\partial x} + \frac{\partial v}{\partial y} + \frac{\partial w}{\partial z} \right] \\
            \tau_{{\color{red}x}{\color{blue}y}} &= \tau_{yx} = \mu \left( \frac{\partial {\color{blue}v}}{\partial {\color{red}x}} + \frac{\partial {\color{red}u}}{\partial {\color{blue}y}} \right) \\
            \tau_{{\color{red}x}{\color{blue}z}} &= \tau_{zx} = \mu \left( \frac{\partial {\color{blue}w}}{\partial {\color{red}x}} + \frac{\partial {\color{red}u}}{\partial {\color{blue}z}} \right) \\
            \tau_{{\color{red}y}{\color{blue}z}} &= \tau_{zy} = \mu \left( \frac{\partial {\color{blue}w}}{\partial {\color{red}y}} + \frac{\partial {\color{red}v}}{\partial {\color{blue}z}} \right)
        \end{align*}
        \item Stokes Hypothesis: $\lambda = -\frac{2}{3} \mu$ (good approximation for gases)
        \item Kinematic viscosity: $\nu = \frac{\mu}{\rho}$
    \end{itemize}
\end{itemize}



\colorbox{black}{\textbf{\color{white}Navier-Stokes Equation / Momentum Equation}}
\begin{itemize}
    \item The following form is for a \textbf{\color{red}Newtonian fluid} reaching \textbf{\color{orange}steady state (constant fluid properties)} in the \textbf{\color{teal}absence of body forces}:
    \begin{align*}
        \underbrace{\frac{D{\color{red}u}}{Dt}}_{\text{accel.}} &= \underbrace{\frac{\partial {\color{red}u}}{\partial t}}_{\text{local accel.}} + \underbrace{u \frac{\partial {\color{red}u}}{\partial x} + v \frac{\partial {\color{red}u}}{\partial y} + w \frac{\partial {\color{red}u}}{\partial z}}_{\text{advection}} \\
        &= - \underbrace{\frac{1}{\rho}\frac{\partial p}{\partial {\color{red}x}}}_{\text{pressure gradient}} + \underbrace{\nu \left[ \frac{\partial^2 {\color{red}u}}{\partial x^2} + \frac{\partial^2 {\color{red}u}}{\partial y^2} + \frac{\partial^2 {\color{red}u}}{\partial z^2} \right]}_{\text{diffusion}} \\
        &= \frac{1}{\rho}\frac{\partial p}{\partial {\color{red}x}} + \nu \nabla^2 {\color{red}u} \\
        \\
        \frac{D{\color{red}v}}{Dt} &= \frac{\partial {\color{red}v}}{\partial t} + u \frac{\partial {\color{red}v}}{\partial x} + v \frac{\partial {\color{red}v}}{\partial y} + w \frac{\partial {\color{red}v}}{\partial z} \\
        &= - \frac{1}{\rho}\frac{\partial p}{\partial {\color{red}y}} + \nu \left[ \frac{\partial^2 {\color{red}v}}{\partial x^2} + \frac{\partial^2 {\color{red}v}}{\partial y^2} + \frac{\partial^2 {\color{red}v}}{\partial z^2} \right]\\
        \\
        \frac{D{\color{red}w}}{Dt} &= \frac{\partial {\color{red}w}}{\partial t} + u \frac{\partial {\color{red}w}}{\partial x} + v \frac{\partial {\color{red}w}}{\partial y} + w \frac{\partial {\color{red}w}}{\partial z} \\
        &= - \frac{1}{\rho}\frac{\partial p}{\partial {\color{red}z}} + \nu \left[ \frac{\partial^2 {\color{red}w}}{\partial x^2} + \frac{\partial^2 {\color{red}w}}{\partial y^2} + \frac{\partial^2 {\color{red}w}}{\partial z^2} \right]
    \end{align*}
    Combining equations in the three directions give the succinct form of:
    \begin{align*}
        \frac{D\bm{V}}{Dt} &= -\frac{1}{\rho} \nabla p + \nu \nabla^2 \bm{V}\\
        \rho \frac{D\bm{V}}{Dt} &= - \nabla p + \mu \nabla^2 \bm{V}
    \end{align*}
    \item For \textbf{\color{orange}steady flow}, the local acceleration terms are zero. The NS equation can be reduced to
    \begin{equation*}
        \rho (\bm{V}\cdot \nabla)\bm{V} = -\nabla p + \mu \nabla^2 \bm{V}
    \end{equation*}
    \item Meaning of each terms:
    \begin{itemize}
        \item \textbf{\color{red}Local acceleration / Local derivative}: 
        \begin{itemize}
            \item $\frac{\partial u}{\partial t}$, $\frac{\partial v}{\partial t}$, $\frac{\partial w}{\partial z}$
            \item time-dependent unsteady term
        \end{itemize}
        \item \textbf{\color{teal}Advection / Convection term}
        \begin{itemize}
            \item e.g. $u \frac{\partial u}{\partial x} + v \frac{\partial u}{\partial y} + w \frac{\partial u}{\partial z}$
            \item Represent inertial force of the fluid flow
            \item $u \frac{\partial u}{\partial x} = $ acceleration due to motion along x-axis
            \item $v \frac{\partial u}{\partial y} = $ acceleration due to motion along y-axis
        \end{itemize}
        \item \textbf{\color{blue}Diffusion term}
        \begin{itemize}
            \item e.g. $\nu \nabla^2 \bm{V}$
            \item Represent diffusion of fluid properties (in this case, the viscous force)
        \end{itemize}
        \item \textbf{\color{olive}Source term}
        \begin{itemize}
            \item Generation and dissipation term for the transport properties
            \item Since we assume that there is not acting body force, the momentum of the fluid is driven by pressure gradient
        \end{itemize}
    \end{itemize}
    \item Number of unknowns:
    \begin{itemize}
        \item There are 4 equations (3 equations from Navier-Stokes for velocity components and 1 from Continuity Equation)
        \item 6 unknowns: 
        \begin{itemize}
            \item Three velocity components: $u$, $v$, $w$;
            \item density: $\rho$;
            \item viscosity: $\nu$;
            \item thermodynamic pressure: $p$
        \end{itemize}
    \end{itemize}
\end{itemize}

\colorbox{black}{\textbf{\color{white}Assumptions Enabled by Incompressible Flow}}
\begin{itemize}
    \item A incompressible flow is defined as the one in which $\nabla \cdot \bm{V} = 0$, which is equivalent to $\frac{D\rho}{Dt} = 0$
    \item \textbf{\color{teal}It is not necessarily true} that $\frac{D\rho}{Dt}=0$ implies both $\frac{\partial \rho}{\partial t}=0$ and $\nabla \rho = 0$ independently. $\frac{D\rho}{Dt}=0$ Only implies that the sum of these two terms are zero. 
    \begin{itemize}
        \item An example: Consider $\rho(x,t)=x-at$ where $a = \frac{\partial x}{\partial t}= u$.
        \begin{align*}
            \frac{D\rho}{Dt} &= \frac{\partial \rho}{\partial t} + \bm{V}\cdot \nabla \rho \\
            &= -a + \begin{bmatrix}
                u \\
                {\color{red}\cancelto{0}{\color{black}v}} \quad \\
                {\color{red}\cancelto{0}{\color{black}w}} \quad
            \end{bmatrix} \cdot \begin{bmatrix}
                \frac{\partial \rho}{\partial x} \\
                \frac{\partial \rho}{\partial y} \\
                \frac{\partial \rho}{\partial z} 
            \end{bmatrix} \\
            &= -a +u \cdot 1 \\
            &= -a + a =0
        \end{align*}
    \end{itemize}
    \item For a \textbf{\color{red}homogeneous} incompressible flow, which means that it has constant density throughout, i.e. $\rho =$ constant, we have $\frac{\partial \rho}{\partial t}=0$ and $\nabla \rho = 0$.
\end{itemize}
